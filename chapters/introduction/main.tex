\chapter{Introduction}
\label{ch:introduction}

%\dictum[Immanuel Kant]{%
  %Hello world }%
%\vskip 1em

\section{Rationale}

\subsection{Relevance of glacial outburst floods}

%Outburst flood definition

%Glacial outburst floods (GOFs) pose significant hazards to communities living in mountainous regions worldwide.

%Glacier floods have directly caused at least: 7 deaths in Iceland, 393 deaths in the European Alps, 5745 deaths in South America and 6300 deaths in central Asia.

%Present global deglaciation is increasing the number and extent of glacial lakes around the world (e.g. Paul et al., 2007; Gardelle et al.,2013; Wang et al., 2011; Carrivick and Tweed, 2013; Carrivick and Quincey, 2014; Tweed and Carrivick, 2015). (AND Zhang 2024?)

%Efforts to mitigate GOFs risks require a comprehensive understanding of the processes driving these events, from their formation to their mode of drainage.

% the following citation contextualize my thesis:

%\cite{Zhang&al2024} in abstract "Future research should prioritize acquiring field data on lake and dam properties"

%\cite{Carrivick&Tweed2016} in abstract: "We recommend that accurate, full and standardised monitoring, recording and reporting of glacier floods is essential if spatio-temporal patterns in glacier flood occurrence, magnitude and societal impact are to be better understood."

\subsection{Motivation of the thesis}

\begin{itemize}
    \item in 2019 Lac des Faverges is artificially drained
    \item the same year WPOF at Trift
    \item We need to improve our understandings on GOFs
\end{itemize}



\section{State of the art of glacial outburst floods research}

In the section we define all type of outburst floods and at the end we convince the reader that some GOFs are a mystery (like WPOFs), or poorly predictable (supra glacial lakes). We are going to study those. 

Glacial lakes review: https://www.sciencedirect.com/science/article/pii/B9780128182345000572

%in the link above:
%Blocking of subsurface outflow channels by sediments, deposits from the mass movement, freezing, or an earthquake-induced settlement of dam structure can increase the lake water level and dam rupture (due to the increasing hydrostatic pressure). Alternatively, the lake level increases until the dam is overtopped and possibly incised and breached. This trigger has rarely been documented. An example is the moraine dam failure of Lake Zhangzhanbo in 1981 (Ding and Liu, 1992; Yamada, 1998).

\subsection{Societal and ecological impacts of GLOFs}

\subsection{GLOFs from ice-dammed lakes (also called jökulhaups)}



\subsection{GLOFs from moraine-dammed lakes}

review: Neupanne 2019
mass movement into proglacial lakes: Haeberli 2017

\subsection{GLOFs from artificial supraglacial lake drainage}

\subsection{WPOFs overview}

% note coming from WPOFs paper writing:

%Interesting: Veh et al (2022) reported 312 (219 GLOFs) for the Swiss Alps (from 1560 to 2019), we report 92 WPOF. Hence, 30% of the glacier flood (n=312) are WPOF, so Haberli's statement that 30/40& of glacier flood are from water pocket stands.

%some consideration of subglacial lakes found in Cappy et al (2010):
%We use the term ‘glacier-dammed lake’ in reference to any lake that is dammed by glacier ice, irrespective of its position relative to the glacier (Reference Blachut, Ballantyne and HamiltonBlachut and Ballantyne, 1976). In contrast, a ‘subglacial lake’ is one that occurs primarily underneath a glacier, whether or not it is at atmospheric pressure (Reference Clague and EvansClague and Evans, 1994;Reference Tweed and RussellTweed and Russell, 1999). This primarily morphological definition contrasts with the usages of Reference SiegertSiegert (2000), who states that subglacial lakes are ‘discrete bodies of water that lie at the base of an ice sheet between ice and substrate’, and Reference HodgsonHodgson and others (2009), who further limit the term to lakes sealed off from, and presumably at higher pressure than, the atmosphere.

%Lake impounded at the glacier forefield, i.e.\ proglacial lakes, are dammed by a moraine and drain when the moraine fails because of the lake hydrostatic pressure \citep{Neupane&al2019}. Lakes at the surface, i.e.\ supraglacial lakes, are most common in cold ice and usually drain when the lake water oversflows the ice dam \citep{Walder&Costa1996,Raymond&al2003,Kingslake&al2015}. Supraglacial lakes can also drain through sudden hydro-fracture propagation \citep[e.g.][]{Christoffersen&al2018}. Lakes at the glacier margins, i.e.\ ice-marginal lakes and lakes at the glacier base, i.e.\ subglacial lakes, are dammed by the ice until that the hydrostatic pressure of the lake water reaches the ice dam flotation and leads to the drainage of the reservoir \citep[e.g.][]{Bjornsson2010}. The drainage can also be initiated through pre-existing veins and channels that are progressively enlarged \citep{Nye1976}. Note that subglacial lakes are usually caused by heat fluxes at the glacier base and are more common in ice sheets or in ice caps associated to geothermal activities, rather than in Alpine glaciers \citep{Livingstone&al2022}. 

% additional info not in the article:

%Historically the naming water pocket was not always used for the origin of the Tête Rousse outburst: \cite{Froidevaux1892} mention an "englacial lake" and \cite{Vallot1894} a water-filled crevasse (note that \cite{Vallot1894} proposed a different mechanism than \cite{Vincent&al2010b}).

% Swift 2021 mentioned "WATER POCKET" reported in Hooke and pohjola 1994 (but the naming water pocket is actually only mentioned in the intro Swift 2021, Hooke and Pohjola speak about scattered constriction).
%\cite{Roberts2005} list the various mode of glacial lake drainage and differentiate "drainage of an intraglacial cavity" from "drainage of subglacial lakes" (see its Fig.1). The author refers to water-filled vault or water-filled subglacial cavities for what \cite{Haeberli1983} call water pockets. Note that the author consider the subglacial linked-cavity system \citep{Kamb1987} to be water-filled subglacial cavities too.

%\cite{Fountain&Walder1998} (the reference review for water in temperate glacier) uses water filled pocket naming for decimeter-scale features (also called voids, or bubbles in \cite{Raymond&Harrison1975}). The authors give some order of magnitude of voids (i.e.\ , potential water accumulation) in ice ($\approx1\%$).% They state that the englacial water storage may exceed the subglacial water storage in temperate glacier (see arguments in text). But the form of the englacial storage and the size of the water inclusions are not well constrained. In their review, only three studies are cited about englacial and subglacial outburst \citep{Haeberli1983,Driedger&Fountain1989, Walder&Driedger1995}, which show the need for more investigations on the topic.
%For instance, \cite{Rounce&al2017} observed an outburst flood on Lhotse Glacier (Everest area, Nepal) in which most of the flood water was stored englacially and in which the flood was triggered by dam failure. The flood's peak discharge was estimated to be 210\,m$^3$\,s$^{-1}$.
%\cite{Byers&al2022} claim that "Less attention has been given to other cryospheric flood phenomena, which include floods sourced primarily from englacial conduits, permafrost-linked rockfall and avalanches, and earthquake-triggered glacial lake floods". The authors call these glaciers outburst "Englacial conduit floods". It seems quite classic in the Khumbu region (see their Figure~4). They discuss about water-filled ice lenses at the glacier/debris cover interface.

%The biggest events are Tête Rousse in 1892, Ferpècle in 1943 and Vadrec d'Albigna in 1942
%Here we don't consider the flow itself after the rupture. Note that the specificity for WP and ice-dammed lake outburst floods  is that flow can  have a high content in ice. Reverse to mud/debris flow is that ice can fracture, can melt during the flow and ice density makes a segregation (ice tends to flow the surface while debris tends to locate at the bed. a singular rheology can be expected. \forauthor{Was this actually ever investigated ?}
%Sediment transport implication: rare event but high magnitude of bed erosion. \cite{Beecroft1983,Swift&al2021} did measure the sediment transport during WPOFs.

%Whilst water pocket formation and filling mechanism are poorly understood, one expect water pocket rupture to occur when the water pressure reach a maximum, leading to the ice dam break by similar mechanisms than GLOFs, i.e.\  by ice dam flotation \citep{Bjornsson2010} and/or by progressive glacial channels enlargements \citep{Nye1976}. 
%The latter is possible due to frictional heating (i.e thermal energy dissipation in the waterflow due to potential energy release) and/or due to sensible heat fluxes (i.e advection of warm water from the lake)
%In general, the latter lead to progressively rising discharge, whereas the former often results in a very fast drainage onset and high discharge \citep{Roberts2005}. 

%snow precip from the MTO analysis:
%5 snow event in the weak to moderate precip. category, and 1 snow event (55mm) in the "heavy precip." catergory

%An outburst due to channel blockage is qualitatively described in \cite{Rabot1905} at Glacier de Savoy, France, in 1899. 

%\cite{VanderVeen2007} investigate the effect of water inflow on crevasse propagation. The authors based their study on the linear elastic fracture mechanics. They find that rapid transfer of surface meltwater to the bed of a cold glacier requires abundant ponding at the surface to initiate and sustain full thickness fracturing before refreezing occurs

%The volume of this water-filled crevasse at Marmolada is estimated to be 18500\,$\pm$\,5500\,m$^3$.

%Tete Rousse: 
%The monitoring of the cavity filling revealed a strong relationship between measured surface melting and the filling rate, with a time delay of 4–6 hours \citep{Vincent&al2015}. However, note that this filling rate is measured after the artificial drainage of 2010, when the cavity has refilled. Before 2010, when the cavity was full of water, one assume there is no water circulation within the cavity because there is no exit for water to flow. 

% There are two types of polythermal glaciers in the Alps: (1) cold firn formation (>4000 masl) with ice transported to the ablation area (Gornergletscher only documented case). (2) in situ cooling of stagnant small glaciers (no crevasses) losing their firn cover. Polythermal conditions for many of these have been modelled (Huss&Fischer 2016). Measurements confirm this for Sex Rouge (Fischer, 2018) and Corvatsch (next to Vadret Alp Ota)

% exemple of speculative events for WPOFs:

%For instance, \cite{Rabot1905} mentions "glacier-pocket pool" to describe the origin of an outburst flood at Glacier de l'A Neuve in June 1898 and Festigletscher in August 1899 (both in Switzerland), and "pool of water in the interior of the glacier" for the origin of a flood at Hobärggletscher in August 1898 (Switzerland), although there is no visual evidence of the so-called glacier-pocket pool in any of these cases. %\cite{Rabot1905} also mention "glacier-pocket reservoir" (literally translated from the french wording "Pôche d'eau") for an outburst flood at Jostedalsbrae in 1904 (Norway), again without clear evidence of the so-called reservoir. \cite{Forel&al1899} used the term water pocket for the outburst at Glacier de l'A Neuve because it is a common term according to the authors, but they claim that this term makes little sense because the water should be trapped in already existing voids (crevasses, fractures, channels, etc...) and not in a pocket created by the water itself, as the name would suggest. Note that this is the earliest mention of "water pocket" we found in the literature.

\section{Research questions}


\begin{itemize}
   \item This thesis explores rare GOFs that are poorly monitored and thus not well known: 1) SLOFs and 2) WPOFs
    \item This thesis provide unique observation on a SLOF, and improve our understanding on WPOFs 
    \item The outcomes of this thesis constitute a basis for further hazard mitigation in the glacierized regions of the world.
    \item  what are the drivers controlling the drainage mode? What are the parameters necessary for numerical modelling of such drainages?
    \item What are \textit{exactly} the water pockets responsible for 30/40\% of the GLOFs in Switzerland? What are their mechanisms of formation and rupture? Can we predict WPOFs?
    \item  How can we detect water pocket in the Alps? By using GPR. What are the limitations?
\end{itemize}

\section{Thesis structure}


\begin{enumerate}
\item Chapter~\ref{ch:chapter_plainemorte}
\item Chapter~\ref{ch:chapter_WPOFs}
\item Chapter~\ref{ch:chapter_gprmax}
\item Discussion and outlook for all Chapters
\end{enumerate}


