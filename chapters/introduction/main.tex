\chapter{Introduction}
\label{ch:introduction}

%\dictum[Immanuel Kant]{%
  %Hello world }%
%\vskip 1em

\section{Rationale}

\subsection{Relevance of glacial outburst floods}
% here we define GLOFs briefly, we say why it matters (=dangerous)(we also refer to the next subsection on societal impacts for more details, or maybe we detail here? in Gilia's thesis this is a short section!), we say why we want to understang the physics of GLOFs and what's needed: a better parameters characterization for modelling, and a betetr undertsanding on the formation/rupture processes, especially on water reservoir that are rarely observed (this thesis).
% at the beginning of the section we know nothing
% at the end we knows what are GLOFs, why it matters, and why we are going to study those

%Glacier lakes definition
Glacial lakes review: https://www.sciencedirect.com/science/article/pii/B9780128182345000572

Glaciers have been losing mass since the end of the Little Ice, and at high — and accelerating — rates since the 1960s \cite{Huggonet2022,Zemp2019}., producing more and larger glacial lakes \cite{Zhang&al2024}. More than 110,000 glacial lakes currently exist, covering a total area of ~15,000 km2, having increased in area by $\approx$22\% from 1990 to 2020
% in veh 2022: Most of the growth in glacier lake area has been tied to moraine-dammed lakes (Buckel et al., 2018; Emmer et al., 2020; Rick et al., 2022; Wilson et al., 2018; Zheng, Allen, et al., 2021),

When the resultant meltwater is naturally dammed (by ice, a moraine or bedrock), glacial lakes can form on (supraglacial), within (englacial), under (subglacial), at the edge of (ice-marginal) or beyond the front of (proglacial) glaciers. %from zhang 2024

There are an estimated 14,394 of these lakes (>0.05 km2) globally \cite{Shugar&al2020}, the majority of which are in Greenland, the Southern Andes, Antarctica and the Subantarctic, Alaska, Canada and High Mountain Asia (HMA).

Currently, glacial lakes hold about 0.43mm of sea level equivalent (Shugar 2020) -> sow we don't care too much on the SLR impact, but rather to the natural hazard associated to.


%Outburst flood definition
Glacier lake outburst floods (GLOFs) represent one of the most catastrophic natural hazards in high mountain regions, posing significant threats to both human lives and infrastructure.

Indeed, more than 10 million people are exposed to the impacts of GLOFs, commonly associated with dam failure or wave overtopping associated with mass movements \cite{Zhang&al2024}.

GLOFs are described as low frequency, high-magnitude events with major geomorphic consequences

Recorded GLOFs have increased in most glaciated mountain regions of the world, with ongoing deglaciation and lake expansion expected to increase GLOF frequency further \cite{Zhang&al2024}.

Glacier floods-related disasters in the Alps caused hundreds of fatalities in the 19th century \cite{Haeberli1983,Vincent&al2010b,Ancey&al2019} possibly motivating the highest glaciological research activity in our study regions \cite{Veh&al2022}.

\cite{Carrivick&Tweed2016}: Of 1348 recorded glacier floods, 24\% also had a societal impact recorded.

\cite{Carrivick&Tweed2016}: Present global deglaciation is increasing the number and extent of glacial lakes around the world (e.g. Paul et al., 2007; Gardelle et al.,2013; Wang et al., 2011; Carrivick and Tweed, 2013; Carrivick and Quincey, 2014; Tweed and Carrivick, 2015). (AND Zhang 2024?)

Given that millions of people are potentially at risk from GLOFs worldwide, especially in a warming climate, understanding the location, timing of formation, evolution and physical characteristics of glacial lakes is critical to mitigating the downstream impacts to communities and infrastructure.

Good terminology and previous reviews in \cite{Korup&Tweed2007} ! To be read. Also some info on englacial/subglacial outbursts

%maybe describe the different reservoir here? and only the GLOFs physics and the next sections?

%Glacial outburst floods (GOFs) pose significant hazards to communities living in mountainous regions worldwide.
% here we narrow down to the Alps, in particular Switzerland
Glofs are the most susceptible in Switzerland!
In \cite{Carrivick&Tweed2016}: "The global impact of glacier outburst floods can be crudely assessed
using the number of events recorded per country and per major world
region (Fig. 8A). Using this measure, north-west America (mainly Alaska),
closely followed by the European Alps (mainly Switzerland) and
Iceland are themost susceptible regions to glacier floods (Fig. 8A).However,
since many floods occur repeatedly from the same location, an assessment
of the global impact should also consider the number of sites
recorded to be affected by glacier floods, per country and per major
world region (Fig. 8B). Given these conditions the European Alps is
the most susceptible region, and Switzerland is the most susceptible
country (Fig. 8B)."



%Efforts to mitigate GOFs risks require a comprehensive understanding of the processes driving these events, from their formation to their mode of drainage.

% the following citation contextualize my thesis:

\cite{Zhang&al2024} in abstract: Future research should prioritize acquiring field data on lake and dam properties, producing globally coordinated multitemporal lake mapping, and robust and efficient modelling of GLOFs for comprehensive hazard assessment and response planning.

\cite{Carrivick&Tweed2016} in abstract: "We recommend that accurate, full and standardised monitoring, recording and reporting of glacier floods is essential if spatio-temporal patterns in glacier flood occurrence, magnitude and societal impact are to be better understood."

\cite{Veh&al2022} in abstract: We invite future research to learn more about the physical drivers of regional GLOF trends to improve projections of GLOF occurrence under ongoing atmospheric warming.
% -> Ok, this is not what we do directly, but we set some basis to go in that direction (with the WPOF inventory at the Swiss scale). That said, we don't conclude on any trend and potential trend driver.

\subsection{Motivation of the thesis}

\citet{Emmer&al2022} identify in four thematic areas: (i) understanding GLOFs – timing and processes; (ii) modelling GLOFs and GLOF process chains; (iii) GLOF risk management, prevention and warning; and (iv) human dimensions of GLOFs and GLOF attribution to climate change. This thesis aime at improving the first theme " understanding GLOFs – timing and processes"

\begin{itemize}
    \item in 2019 Lac des Faverges is artificially drained
    \item the same year WPOF at Trift
    \item We need to improve our understandings on GOFs
\end{itemize}

On why we aim to inevntory WPOFs: (from Emmer 2022) More comprehensive GLOF inventories are essential for better understanding frequency of GLOF occurrence in changing mountain environments (Veh et al., 2019; Emmer et al., 2020) and for revealing frequency–magnitude relationships (Hewitt, 1982; Haeberli, 1983), as well as for GLOF attribution to anthropogenic climate change (Harrison et al., 2018; see also Sect. 4.7).



\section{State of the art of glacial outburst floods research}

\citet{Emmer&al2022}: Our analysis of 594 GLOF papers published in 2017–2021 revealed that (i) the number of published GLOF papers experienced a sharp rise (110\% more papers in 5 years)

In the section we define all type of outburst floods and at the end we convince the reader that some GOFs are a mystery (like WPOFs), or poorly predictable (supra glacial lakes). We are going to study those. 

Glacial lakes review: https://www.sciencedirect.com/science/article/pii/B9780128182345000572

%in the link above:
%Blocking of subsurface outflow channels by sediments, deposits from the mass movement, freezing, or an earthquake-induced settlement of dam structure can increase the lake water level and dam rupture (due to the increasing hydrostatic pressure). Alternatively, the lake level increases until the dam is overtopped and possibly incised and breached. This trigger has rarely been documented. An example is the moraine dam failure of Lake Zhangzhanbo in 1981 (Ding and Liu, 1992; Yamada, 1998).

\subsection{Physical and societal impacts of GLOFs}% and ecological?
% or is it more a section on the global aspect of GLOFs research?


% chronological history of GLOFs research
First comprehensive global assessment of the impacts of glacier outburst floods on communities and economies by \cite{Carrivick&Tweed2016}. Glacier floods have directly caused at least: 7 deaths in Iceland, 393 deaths in the European Alps, 5745 deaths in South America and 6300 deaths in central Asia.

For instance, the 1941 GLOF in Huaraz, Peru, destroyed a third of the city and killed about 5,000 people (ref 36 in zhang et al), and the 2013 GLOF in Kedarnath, India, killed more than 6,000 people (ref 37 in zhang et al).

Table 1 in \cite{Carrivick&Tweed2016} compile Key data sources used for the compilation of physical and societal impact attributes of glacier outburst floods. Other major sources that were not region-specific included -> Maybe I copy past in my thesis and update to 2024 (that would highlight the increase of global study since 2016 \citep[e.g][]{Livingstone&al2022,Veh&al2022,Zhang&al2024}


\cite{Carrivick&Tweed2016}: Overall they compiled records of 1348 glacier floods (Fig. 1; Table 2). This is the biggest single compilation of the occurrence and characteristics of glacier floods to date.
\cite{Emmer&al2022}: This study offers an overview of recent GLOF research by analysing 594 peer-reviewed GLOF studies published between 2017 and 2021
% the two studies cited above do not quantify the observations biases, excepted to be large. 
\cite{Veh&al2022}: global inventory of >2800 events (until March 2022). "Here we estimate trends and biases in GLOF reporting based on the largest global catalog of 1,997 dated glacier-related floods in six major mountain ranges from 1901 to 2017" %This studies indicate that GLOFs may be more frequent than previously thought
\cite{Lutzow&al2023}: update of GLOFs global inventory: More than 3,000 GLOFs have been reported worldwide from 850 to 2022.
\cite{Zhang&al2024}: Although data limitations are substantial, more than 3,000 GLOFs have been recorded
from 850 to 2022 \citep{Lutzow&al2023}, particularly in Alaska (24\%), High Mountain Asia (HMA; 18\%) and Iceland (19\%), the majority (64.8\%) being from ice-dammed lakes. %wait: it this ratio from Zhang or from Lutzow?
\cite{Zhang&al2024}'s review focus on the physical aspects of glacial lakes and GLOFs in a warming world.

Fun note: a positive aspect of glofs: for example tens of glacier floods in Norway were noted to have contributed
additional water into hydropower reservoirs (Jackson and
Ragulina, 2014).

\cite{Emmer&al2022}: Hence, the geographical focus of research appears to be driven by potential societal impacts and relevance of GLOFs and the size of these mountain regions, rather than by the physical GLOF processes themselves.

% good example on the ‘word-of-mouth’ reports of glacier floods which are difficult to substantiate in \cite{Carrivick&Tweed2016}: for example Vivian (1979) was told that several thousand people were killed when a huge flood was generated fromice fall into a proglacial lake in Tibet (see Tufnell, 1984).



\subsection{Our physical understanding of GLOFs}
% careful to no mix up things with the subsection modelling (otherwise merge)

% message: while global studies increase thanks to increase in temporal and spatial observational means (e.g. remote sensing, global inventories, see subsection before on global inventory), physical uinderstanding still need to be improved for accurate GLOFs modelling.


% Historical work by Nye, Rothlisberger et companie (read again Bjornsson 2010 and summarize here)




%\cite{Emmer&al2022}: A pronounced trend in understanding the occurrence of GLOFs from a large scale perspective (mountain ranges, large regions) is the building of updated GLOF inventories, typically revealing the incompleteness of existing GLOF records (Table 2). %does it belong to here or to a more "global GLOFs" subsection?
Still in Emmer et al 2022: This trend is associated with increasing availability and resolution of satellite images (Kirschbaum et al., 2019; Taylor et al., 2021), allowing detailed analysis of persistent geomorphic GLOF diagnostic features, both in a manual and in a semi-automatic way (Veh et al., 2018).

In \cite{Emmer&al2022}: Numerous studies not only describe, analyse and model the hydrodynamics and geomorphological imprints of GLOFs (e.g. Clague and Evans, 2000; Emmer, 2017; Jacquet et al., 2017) but also assess pre-GLOF conditions and hazard drivers (climatological, glaciological, geological) and eluci date plausible GLOF scenarios (e.g. Carrivick et al., 2017; Haeberli et al., 2017; Mergili et al., 2020; Klimeš et al., 2021; Zheng et al., 2021b; Emmer et al., 2020, 2022).

In \cite{Emmer&al2022}: Simulations of GLOF process chains have been performed since the early 2000s, and at least three stages of research evolution can be distinguished:
% empirical mass point models
1. relatively simple empirical mass point models such as MSF (Huggel et al., 2003, 2004), mainly suited for regional-scale applications and still applied more recently at such scales (r.randomwalk and its predecessors – Gruber and Mergili, 2013; Mergili et al., 2015);
% advances model chain
2. more advanced model chains, applying tailored physically based simulation tools for each component of the process chain and coupling them at the process boundaries (Schneider et al., 2014;Worni et al., 2014; Schaub et al., 2016);
% three phases mass flow model
3. the emergence of two-phase (Pudasaini, 2012) and later three-phase (Pudasaini and Mergili, 2019) mass flow models and related simulation tools (Mergili et al., 2017; Mergili and Pudasaini, 2021) and the trend moving towards integrated simulations, considering the entire GLOF process chain in one single simulation step.

For the latter stage: one need to define the GLOFs scenario, and exploring the field of tension between the physical details and practical applicability of the available simulation tools(i.e. -> often need parametrisation (link to PM chapter :)

%Required but hard-to-obtain modelling parameters for erosion and deposit summarized in Emmer 2022 (see Table 3)

In Emmer 2022: With the growing need for hazard assessments in areas with limited access, where potential GLOF exposure in the downstream regions is high (Allen et al., 2016, 2019; Schwanghart et al., 2016), predictive GLOF modelling serves a purpose (e.g. Sattar et al., 2019a, b, 2021). However, such modelling demands prior evaluation of the breach parameters such as breach depth, breach width and the breach formation time. These parameters are difficult to estimate and depend on multiple factors such as the nature of the
damming material, the trigger event (e.g. avalanche, land slide, internal moraine failures), the nature of the impact wave, freeboard of the lake and lake bathymetry. There fore, one must rely on empirical methods or scenario definition as alternatives to determine these parameters exactly.

Table 3 in Emmer 2022 show a: "Summary of the key topics, observed trends, identified challenges and proposed ways forward in the thematic area modelling GLOFs and GLOF process chains.". In "identified challenges" there is "Obtaining (in situ) parameters required by the model (e.g. breach parameters, lake bathymetry) and addressing their uncertainties (e.g. Schaub et al., 2016; Mergili et al., 2018b, 2020; Sattar et al., 2020; see Sect. 4.4)" and: "Defining realistic GLOF scenarios in predictive GLOF modelling". That makes another link to our thesis (especially to our PM chapter: which parameters for supra drainage? Which mode: stable or unstable?). 
This table also propose ways forward such as: "Extending a set of detail-modelled events in order to obtain a plausible range of parameters for predictive modelling (e.g. Mergili et al., 2018a, b, 2020). That is also what ou PM study does.
IMPORTANT: I think here the author refer to "moraine breaching" parameter, but I assume you also want to know how the ice dam breach, right? Costa 1985 mention breaching of ice dam at least

\subsubsection{GLOFs from moraine-dammed lakes}

review: Neupanne 2019
mass movement into proglacial lakes: Haeberli 2017
Harrison et al. (2018) observed a increasing trend in outbursts from moraine-dammed lakes, possibly because of a lagged response of GLOF occurrence to climate forcing, glacier retreat and lake formation. However, the number of GLOFs recorded may
be underestimated in some regions because of low research
monitoring. %what's Veh is saying about this?

\subsubsection{GLOFs from ice-dammed lakes (also called jökulhaups)}
% I guess this accounts for supraglacial lakes too!!
% and I also assume WPOFs are counted in too -> make a good distinction/sub-category

The most present \citep{Zhang&al2024} 64.8\,\%. Partly owing
to the recurrence of multiple GLOFs from individual ice-dammed lakes (I guess a moraine break empty the lake at all).

Ice-dammed lakes, in contrast to moraine-dam lakes, are more short-lived, and their formation is tied to thick glaciers that can impound lakes at their margins, a common situation along glaciers descending from ice fields in Alaska (Field et al., 2021; Rick et al., 2022) and Patagonia (Wilson et al., 2018). %but also the Alps! cite gornersee Huss 2007, Ancey 2019 Gietroz, Marjelensee in Altesch.

%Trend? see temporal

\subsubsubsection{GLOFs from artificial supraglacial lake drainage}

\subsubsubsection{WPOFs overview} % maybe quite small -> what we need to know is that we don't know and that is why we have chapter 2

Wrong appelation at times (exemples?). Similar as GLOFs, example in \cite{Emmer&al2022} "Importantly, proper terminology should be maintained among researchers and also among disaster risk reduction practitioners and authorities in order to avoid misinterpretation of individual events. Many mass flow events are often immediately termed GLOFs because GLOFs have received major attention recently. An example is an early interpretation of the 2021 Chamoli disaster, which was described as a GLOF by some shortly after it happened (e.g. https://indianexpress.com/article/explained/). However, detailed analysis revealed that it originated as a rock and ice avalanche and no lake was involved in the process chain propagation (Shugar et al., 2021).



% note coming from WPOFs paper writing:

%Interesting: Veh et al (2022) reported 312 (219 GLOFs) for the Swiss Alps (from 1560 to 2019), we report 92 WPOF. Hence, 30% of the glacier flood (n=312) are WPOF, so Haberli's statement that 30/40& of glacier flood are from water pocket stands.

%some consideration of subglacial lakes found in Cappy et al (2010):
%We use the term ‘glacier-dammed lake’ in reference to any lake that is dammed by glacier ice, irrespective of its position relative to the glacier (Reference Blachut, Ballantyne and HamiltonBlachut and Ballantyne, 1976). In contrast, a ‘subglacial lake’ is one that occurs primarily underneath a glacier, whether or not it is at atmospheric pressure (Reference Clague and EvansClague and Evans, 1994;Reference Tweed and RussellTweed and Russell, 1999). This primarily morphological definition contrasts with the usages of Reference SiegertSiegert (2000), who states that subglacial lakes are ‘discrete bodies of water that lie at the base of an ice sheet between ice and substrate’, and Reference HodgsonHodgson and others (2009), who further limit the term to lakes sealed off from, and presumably at higher pressure than, the atmosphere.

%Lake impounded at the glacier forefield, i.e.\ proglacial lakes, are dammed by a moraine and drain when the moraine fails because of the lake hydrostatic pressure \citep{Neupane&al2019}. Lakes at the surface, i.e.\ supraglacial lakes, are most common in cold ice and usually drain when the lake water oversflows the ice dam \citep{Walder&Costa1996,Raymond&al2003,Kingslake&al2015}. Supraglacial lakes can also drain through sudden hydro-fracture propagation \citep[e.g.][]{Christoffersen&al2018}. Lakes at the glacier margins, i.e.\ ice-marginal lakes and lakes at the glacier base, i.e.\ subglacial lakes, are dammed by the ice until that the hydrostatic pressure of the lake water reaches the ice dam flotation and leads to the drainage of the reservoir \citep[e.g.][]{Bjornsson2010}. The drainage can also be initiated through pre-existing veins and channels that are progressively enlarged \citep{Nye1976}. Note that subglacial lakes are usually caused by heat fluxes at the glacier base and are more common in ice sheets or in ice caps associated to geothermal activities, rather than in Alpine glaciers \citep{Livingstone&al2022}. 

% additional info not in the article:

%Historically the naming water pocket was not always used for the origin of the Tête Rousse outburst: \cite{Froidevaux1892} mention an "englacial lake" and \cite{Vallot1894} a water-filled crevasse (note that \cite{Vallot1894} proposed a different mechanism than \cite{Vincent&al2010b}).

% Swift 2021 mentioned "WATER POCKET" reported in Hooke and pohjola 1994 (but the naming water pocket is actually only mentioned in the intro Swift 2021, Hooke and Pohjola speak about scattered constriction).
%\cite{Roberts2005} list the various mode of glacial lake drainage and differentiate "drainage of an intraglacial cavity" from "drainage of subglacial lakes" (see its Fig.1). The author refers to water-filled vault or water-filled subglacial cavities for what \cite{Haeberli1983} call water pockets. Note that the author consider the subglacial linked-cavity system \citep{Kamb1987} to be water-filled subglacial cavities too.

%\cite{Fountain&Walder1998} (the reference review for water in temperate glacier) uses water filled pocket naming for decimeter-scale features (also called voids, or bubbles in \cite{Raymond&Harrison1975}). The authors give some order of magnitude of voids (i.e.\ , potential water accumulation) in ice ($\approx1\%$).% They state that the englacial water storage may exceed the subglacial water storage in temperate glacier (see arguments in text). But the form of the englacial storage and the size of the water inclusions are not well constrained. In their review, only three studies are cited about englacial and subglacial outburst \citep{Haeberli1983,Driedger&Fountain1989, Walder&Driedger1995}, which show the need for more investigations on the topic.
%For instance, \cite{Rounce&al2017} observed an outburst flood on Lhotse Glacier (Everest area, Nepal) in which most of the flood water was stored englacially and in which the flood was triggered by dam failure. The flood's peak discharge was estimated to be 210\,m$^3$\,s$^{-1}$.
%\cite{Byers&al2022} claim that "Less attention has been given to other cryospheric flood phenomena, which include floods sourced primarily from englacial conduits, permafrost-linked rockfall and avalanches, and earthquake-triggered glacial lake floods". The authors call these glaciers outburst "Englacial conduit floods". It seems quite classic in the Khumbu region (see their Figure~4). They discuss about water-filled ice lenses at the glacier/debris cover interface.

%The biggest events are Tête Rousse in 1892, Ferpècle in 1943 and Vadrec d'Albigna in 1942
%Here we don't consider the flow itself after the rupture. Note that the specificity for WP and ice-dammed lake outburst floods  is that flow can  have a high content in ice. Reverse to mud/debris flow is that ice can fracture, can melt during the flow and ice density makes a segregation (ice tends to flow the surface while debris tends to locate at the bed. a singular rheology can be expected. \forauthor{Was this actually ever investigated ?}
%Sediment transport implication: rare event but high magnitude of bed erosion. \cite{Beecroft1983,Swift&al2021} did measure the sediment transport during WPOFs.

%Whilst water pocket formation and filling mechanism are poorly understood, one expect water pocket rupture to occur when the water pressure reach a maximum, leading to the ice dam break by similar mechanisms than GLOFs, i.e.\  by ice dam flotation \citep{Bjornsson2010} and/or by progressive glacial channels enlargements \citep{Nye1976}. 
%The latter is possible due to frictional heating (i.e thermal energy dissipation in the waterflow due to potential energy release) and/or due to sensible heat fluxes (i.e advection of warm water from the lake)
%In general, the latter lead to progressively rising discharge, whereas the former often results in a very fast drainage onset and high discharge \citep{Roberts2005}. 

%snow precip from the MTO analysis:
%5 snow event in the weak to moderate precip. category, and 1 snow event (55mm) in the "heavy precip." catergory

%An outburst due to channel blockage is qualitatively described in \cite{Rabot1905} at Glacier de Savoy, France, in 1899. 

%\cite{VanderVeen2007} investigate the effect of water inflow on crevasse propagation. The authors based their study on the linear elastic fracture mechanics. They find that rapid transfer of surface meltwater to the bed of a cold glacier requires abundant ponding at the surface to initiate and sustain full thickness fracturing before refreezing occurs

%The volume of this water-filled crevasse at Marmolada is estimated to be 18500\,$\pm$\,5500\,m$^3$.

%Tete Rousse: 
%The monitoring of the cavity filling revealed a strong relationship between measured surface melting and the filling rate, with a time delay of 4–6 hours \citep{Vincent&al2015}. However, note that this filling rate is measured after the artificial drainage of 2010, when the cavity has refilled. Before 2010, when the cavity was full of water, one assume there is no water circulation within the cavity because there is no exit for water to flow. 

% There are two types of polythermal glaciers in the Alps: (1) cold firn formation (>4000 masl) with ice transported to the ablation area (Gornergletscher only documented case). (2) in situ cooling of stagnant small glaciers (no crevasses) losing their firn cover. Polythermal conditions for many of these have been modelled (Huss&Fischer 2016). Measurements confirm this for Sex Rouge (Fischer, 2018) and Corvatsch (next to Vadret Alp Ota)

% exemple of speculative events for WPOFs:

%For instance, \cite{Rabot1905} mentions "glacier-pocket pool" to describe the origin of an outburst flood at Glacier de l'A Neuve in June 1898 and Festigletscher in August 1899 (both in Switzerland), and "pool of water in the interior of the glacier" for the origin of a flood at Hobärggletscher in August 1898 (Switzerland), although there is no visual evidence of the so-called glacier-pocket pool in any of these cases. %\cite{Rabot1905} also mention "glacier-pocket reservoir" (literally translated from the french wording "Pôche d'eau") for an outburst flood at Jostedalsbrae in 1904 (Norway), again without clear evidence of the so-called reservoir. \cite{Forel&al1899} used the term water pocket for the outburst at Glacier de l'A Neuve because it is a common term according to the authors, but they claim that this term makes little sense because the water should be trapped in already existing voids (crevasses, fractures, channels, etc...) and not in a pocket created by the water itself, as the name would suggest. Note that this is the earliest mention of "water pocket" we found in the literature.


\subsection{Spatio-temporal distribution of GLOFs}

\subsubsection{Temporal}

Globally, the size and abundance of glacier lakes has doubled between 1990 and 2017 (Shugar et al., 2020).
So we can expect a increase in GLOFs, the following whows it's true for GLOF from moraine-dam, but not for the other type (at least no conclusive from my understanding)

\cite{Carrivick&Tweed2016}: The number of recorded glacier floods per time period has apparently reduced since the mid-1990s in all major world regions, but the reasons for this apparent trend are unclear.
-> check \cite{Veh&al2022} for an explaination

Many scientists argue that the annual number of GLOFs might increase as a consequence of atmospheric warming (Bajracharya & Mool, 2009; Bolch et al., 2012; Clague & Evans, 2000; Cook et al., 2018; Harrison et al., 2018; Liu et al., 2014; Miles et al., 2018; Richardson & Reynolds, 2000; Shugar et al., 2020; Tweed & Russell, 1999; Y. Chen, 2010).

Thousands of lakes have been forming at the margins of retreating glaciers (Shugar et al., 2020), with reported outbursts deemed to be on the rise, especially since the beginning of the 20th century (Harrison et al., 2018).

% future global trend
In HMA, GLOF hazards are projected to triple by 2100, but changes in other regions will likely be lower given topographic constraints on lake evolution.

%Impact of anthropogenic forcing on GLOFs

Recent research has established clear causality from greenhouse gas emissions to glacier shrinkage, lake growth and formation and, possibly, to GLOF hazard (Harrison et al., 2018; Huggel et al., 2020a; Stuart-Smith et al., 2021).% check the latter: is GLOFs (and not only lake) impacted by global warming? I guess yes for the proglacial lakes, but for ice-dammed?

%trend for moraine-dam lakes: 
From \cite{Harrison&al2018} abstract: We observe a clear global increase in GLOF frequency and their regularity around 1930, which likely represents a lagged response to post-Little Ice Age warming. Notably, we also show that GLOF frequency and regularity – rather unexpectedly – have declined in recent decades even during a time of rapid glacier recession. Although previous studies have suggested that GLOFs will increase in response to climate warming and glacier recession, our global results demonstrate that this has not yet clearly happened. From an assessment of the timing of climate forcing, lag times in glacier recession, lake formation and moraine-dam failure, we predict increased GLOF frequencies during the next decades and into the 22nd century.
% from harrison still: We argue that response times do not necessarily reflect linear processes and that lake growth may result in none, single or multiple GLOFs from the same lake systems.

%Example of human-induced increase of GLOFs: Stuart-Smith, R. F., Roe, G. H., and Allen, M. R.: Increased outburst flood hazard from Lake Palcacocha due to human-induced glacier retreat, Nat. Clim. Change, 14, 85–90, https://doi.org/10.1038/s41561-021-00686-4, 2021. -> so case studies can demonstrate the increasing of GLOFs hazard associated to climate change. But the same conclusion cannot be done at the global scale (see Veh 2022)

%Veh work:

Temperature influence on GLOFs:
Rising air temperatures could change the meltwater input to lakes, and thus may increase glacier lake volume, given a suitable geometry of the newly exposed basin (Richardson & Reynolds, 2000). In the Himalayas, for example, only 10\% of all glaciers terminate in proglacial lakes; yet some of those lakes may accelerate ice melt and generate more meltwater (King et al., 2019). Warmer air and lake temperatures may thin ice dams and thus lower the threshold to open subglacial drainage channels, likely initiating more GLOFs from glacier-dammed lakes (Clarke, 2003; Kingslake & Ng, 2013; Ng & Liu, 2009). Atmospheric warming might also increase the frequency of moraine-dam failures. Buried ice within a moraine dam can melt, thus reducing cohesion and promoting dam collapse. Gradual settlement of the dam can reduce the lake freeboard and increase the chance for overtopping (Richardson & Reynolds, 2000). Thawing ice and permafrost could also destabilize adjacent glaciers and rockfalls (Haeberli et al., 2017) such that avalanches of ice and rock enter glacier lakes, leading to overtopping and dam breaks (Emmer & Vilímek, 2013; Harrison et al., 2018; Richardson & Reynolds, 2000).

%veh
A simple linear relationship between GLOF occurrences and atmospheric warming thus remains elusive, at least in the past five decades \cite{Veh&al2022}

\cite{Veh&al2022}: We find that the positive trend in the number of reported GLOFs has decayed distinctly after a break in the 1970s, coinciding with independently detected trend changes in annual air temperatures and in the annual number of field-based glacier surveys (a proxy of scientific reporting).
Our analysis emphasizes a distinct geographic and temporal bias in GLOF reporting, and we project that between two to four out of five GLOFs on average might have gone unnoticed in the early to mid-20th century.


\cite{Veh&al2022}: Regardless of model choice, all hindcasts imply that GLOFs have likely been under reported for most of the 20th century outside the European Alps. % so the obs biases is not too strong in the Alps, after all?

%don't forget to narrow down the intro to the Alps at some point


Do we actually have a temporal trend in ice-dammed lakes? I don't think so (just for moraine dam lake yes Harisson 2018). Do we have idea of temporal trend of surface lake or WPOF ? sure not for the latter. For the former: it might be correlated to air warming though, but no study in my knowledge quantify this (Veh 2022 but they did not make a particular case of supraglacial lake?)

%ice-damed lake trend: potentially decreasing because of glacier downwasting (see arguments in veh 2022 below)
% ice-damed lake are the one having the most change in reporting trend (see conclusion in VEh 2022)
With climate-enhanced melting, ice-dammed lakes can enter cycles of outbursts such as lakes dammed by Tulsequah or Kennicott Glaciers that produced multiple GLOFs per year during the 20th century (Geertsema & Clague, 2005; Rickman & Rosenkrans, 1997). Yet several hundreds of ice-dammed lakes have disappeared in our study regions as a consequence of downwasting and retreat of glacier dams in past decades (Geertsema and Clague, 2005; Tweed & Russell, 1999; Wolfe et al., 2014).
\cite{Veh&al2022}: In Alaska, for example, the total number and area of ice-dammed lakes decreased by 13\% and 43\%, respectively, between 1984 and 2019. But note an increase in GLOFs (!) -> the regional increases in ice-dammed failures (Figure 2) are tied to fewer lakes that burst more frequently.
Despite some current positive trends, we may thus see declining trends in ice-dam failures sometime in the future, when ongoing ice loss impedes the formation of such dams (Rick et al., 2022).

% trend in supraglacial lakes
GLOFs originating from bedrock-dammed or supraglacial lakes are assumed to become more frequent with ongoing glacier recession (Emmer et al., 2022; Miles et al., 2018).





\subsubsection{spatial}

\subsection{Monitoring GLOFs}% that is where geophyiscal investigations and in particular GPR fits in

In most cases, mapping relies on the use of the normalized difference water index (NDWI) [ref 40 in zhang 2024] — a satellite image-derived calculation combining green and near-infrared wavelength bands to differentiate water bodies from all other landcover.

In addition to mapping lake outlines, knowledge of glacial lake volume is important for quantifying hazards and for estimating downstream impacts of GLOFs48,49 (Fig. 2). Empirical volume–area relationships for glacial lakes have been established in many instances (see Supplementary Table 1 n Znang 2024). 
Volume and lake area is well correlated (ref 50 in zhang 2024)

\subsection{Modelling of GLOFs}

in \cite{Carrivick&Tweed2016}: "Mindful of these uncertainties in glacier flood attributes [he is referring to uncerttinties in glacier flood volume], it perhaps
seems prudent to consider using empirical hydrograph reconstructions
(Herget et al., 2015) and stochastic simulations of inundation (Watson
et al., 2015). These approaches contrast with the detailed knowledge
needed for mechanistic modelling that preferably relies on lake level
changes or else an input hydrograph, plus down-valley observations
of hydraulics, plus a high-resolution digital elevation model, plus expertise
to run the model (e.g. Carrivick et al., 2009, 2010).Morphodynamic
models of glacier floods, which could be more accurate than hydrodynamic-
only models where there is widespread and intense sediment
transport (e.g. Staines and Carrivick, 2015; Guan et al., 2015), are even
more computationally demanding."

Kingslake and Ng 2013 -> assess modeling performance on a GLOFs in Kyrghistan -> Local predictions of the occurrence and timing of GLOFs from temperature data alone have been only partly successful


\cite{Kingslake&al2015} -> supraglacial modelling (see intro in ogier 2021)


\section{Research questions}

We wrap up the research gaps that we introduced above. 
1) lack of case studies to infer parameters needed for modelisation 
2) lack of knowledge and characterisation observations in rare outburst (need observations)

This lead to the following research question


\begin{itemize}
   \item This thesis explores rare GOFs that are poorly monitored and thus not well known: 1) SLOFs and 2) WPOFs
    \item This thesis provide unique observation on a SLOF, and improve our understanding on WPOFs 
    \item The outcomes of this thesis constitute a basis for further hazard mitigation in the glacierized regions of the world.
    \item  what are the drivers controlling the drainage mode? What are the parameters necessary for numerical modelling of such drainages?
    \item What are \textit{exactly} the water pockets responsible for 30/40\% of the GLOFs in Switzerland? What are their mechanisms of formation and rupture (see also englacial outburst in \cite{Korup&Tweed2007} and other refs) ? Can we predict WPOFs?
    \item  How can we detect water pocket in the Alps? By using GPR. What are the limitations?
\end{itemize}

\section{Thesis structure}


\begin{enumerate}
\item Chapter~\ref{ch:chapter_plainemorte}
\item Chapter~\ref{ch:chapter_WPOFs}
\item Chapter~\ref{ch:chapter_gprmax}
\item Discussion and outlook for all Chapters
\end{enumerate}


