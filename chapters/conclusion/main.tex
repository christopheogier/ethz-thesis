\chapter{Conclusion}
\label{ch:summary}
%\pagestyle{plain}

From the results of this thesis, we could answer the research questions presented in Introduction (Table~\ref{table_GLOFs_research_gaps}). In this final chapter, each research question is reported again, and the main conclusions are summarized.

\section{How can in situ measurements be used to constrain the empirical parameters for flow resistance and heat transfer in the drainage of ice-dammed lakes?}

In Chapter~\ref{ch:chapter_plainemorte}, we conducted extensive field measurements of an ice-dammed lake drainage in a supraglacial channel to characterize the hydraulics and thermodynamics of such an event. Our in situ monitoring network was maintained for more than two months (18 visits in total). The dataset includes data on lake level elevation, lake bathymetry, channel-water height, channel-floor elevation, channel-floor erosion, channel-water temperature, and channel-stream velocity and discharge. This dataset allowed us to calculate the Darcy-Weisbach friction factor ($f_D$), which characterizes flow resistance in the channel, and the Nusselt number ($\Nu$), which characterizes heat transfer between the drained water and the glacier ice.
%

The Darcy-Weisbach friction factor was calculated from the channel geometry, water height, and stream velocity, and averaged for one transect of the channel at seven different points in time. It varied from 0.17 to 0.48, suggesting that the friction factor should be treated as a stochastic variable rather than a constant in modeling studies.
%

The Nusselt number was calculated using four different methods. The first method, the Dittus-Boelter parametrization, used empirical parameters and the Reynolds numbers (obtained from water height and stream velocities). This method provides a continuous time series of Nusselt numbers during the lake drainage for one location in the channel. The second method, the Gnielinski parametrization, used the Darcy-Weisbach friction factor and the Reynolds numbers. This method provided Nusselt numbers at seven points in time averaged for a channel transect. The third method, called the melt-rate method (MR), used water temperature and channel erosion rates at one location to provide Nusselt numbers for six period of time. The fourth method, called the spatial-cooling rate method (SCR), used the decrease in water temperature as a function of distance along the channel and stream discharge. The SCR method provided a continuous time series of Nusselt numbers averaged for a channel transect.
%

The Nusselt numbers obtained by the MR and SCR methods were consistent from each other but differed from those obtained by the Dittus-Boelter and Gnielinski empirical parametrizations. Consequently, we proposed new coefficients for the Dittus-Boelter parametrization that fit better our observations obtained by the SCR method. These discrepancies in the empirical coefficients of the Dittus-Boelter parametrization indicate that these coefficients should also be treated as stochastic variables in modelling studies. 
%In addition, the characteristic length used for calculating heat transfer from the Nusselt numbers should be better constrained, as we also highlighted its influence.

In conclusion, the in situ measurements acquired in this thesis provided the data needed to refine and constrain empirical parameters for flow resistance and heat transfer. Our results highlight differences in these parameters compared to those obtained in other field-based studies. We thus encourage future research to establish a unified modeling framework for ice-dammed lake drainage based on independent field-based datasets of supraglacial streams.


\section{What is the spatial and temporal distribution of outburst floods from water pocket in the Swiss Alps?}

In Chapter~\ref{ch:chapter_WPOFs}, we extended and updated an existing inventory of water pocket outburst floods (WPOFs) in Switzerland, compiling data on 91 events originating from 37 glaciers, including 20 glaciers with repetitive occurrences of WPOFs. Spatially, the majority of documented WPOFs are concentrated in the Bernese and Pennine Alps, particularly notable in the Mischabel group and the Swiss side of the Mont-Blanc massif. Note that these regions are also the most glacierized, making any conclusions on the spatial distribution uninformative. Direct observations of the en-/subglacial water reservoirs are limited to only three WPOFs, highlighting the challenge in detecting and documenting these events. We acknowledge that a substantial number of WPOFs (small or/and remote) were likely unnoticed and are therefore absent from our inventory. 
%

The analysis of the temporal distribution of WPOFs reveals clear seasonal trends, with almost all events (68 out of 75) occurring between June and September. This indicates the primarily role of water inputs from icemelt, snowmelt or rainfall in the formation and rupture of water pockets. However, no clear interannual trends are observed. A linear regression analysis suggests a 5\,\% increase in events per decade since 1900, but the p-value of 0.139 indicates that this trend is not statistically significant at the 0.05 level. We conclude that this observed trend is due to observation bias rather than a true change.


\section{What are the drivers and mechanism for the formation and the rupture of water pockets?}

In Chapter~\ref{ch:chapter_WPOFs}, our analysis of 32 WPOFs with available meteorological data in the Swiss Alps showed unusually high temperatures in the one or two days prior to the event. Additionally, approximately a quarter of these events coincided with strong (>\,10\,mm) to heavy (50\,mm) cumulative precipitation on the day of the event. This indicates that most water pockets formed rapidly within a few days due to significant water input from melting and intense precipitation. These conditions likely destabilize the glacier's drainage system, leading to water pocket rupture and to the outburst flood. In contrast, water pockets forming in polythermal glacier and isolated from the drainage system can accumulate water over longer periods, as in Glacier de Tête Rousse (France), which developed over 40 years.  
%

We compared the glacio-geomorphic variables of glacier having reported WPOF(s) to the glacier having none. These variables are the accumulation area ratio, relative coverage with supraglacial debris, the glacier's average slope, mean aspect, surface area, median elevation, and mean ice surface velocity. Mann-Whitney U tests showed that all variables, except aspect, exhibited statistically significant differences at the 0.05 level between glacier having reported WPOFs and the ones having none. However, the interpretability of this significance is questioned because the observational biases might influence the data and because the number of glaciers with known WPOFs (37) is relatively small compared to the entire glacier population in Switzerland (1400), making robust conclusions difficult. Consequently, we suggest that smaller-scale topographic and glacio-geomorphic variables control the WPOF occurrence, rather than the glacier-wide ones. However, due to limited information on the exact locations of water pockets in our dataset, detailed smaller-scale analyses was not possible. 
%

Based on our inventory of WPOFs in the Swiss Alps and a literature review, we identified four main mechanisms responsible for the formation of water pockets within alpine glaciers: temporary blockage of subglacial channels, hydraulic barriers, water-filled crevasses, and thermal barriers. Temporary blockage occurs when collapsing ice blocks obstruct the subglacial stream, causing water to accumulate until the dam breaks and suddenly release the water stored. Hydraulic barriers form when water accumulates at the glacier bed due to local minima of hydraulic potential, with surrounding ice acting as a temporary dam until pressure forces a rupture through the subglacial drainage network. Water-filled crevasses isolate water from the main drainage system, eventually connecting to the subglacial drainage network through hydrofracturing and enlarging of subglacial channels, which can eventually lead to an outburst. Thermal barriers trap englacial and/or subglacial water at the cold-temperate transition of polythermal glaciers, leading to outburst floods when water pressures build and eventually break the ice dam, or when ice temperatures rise to the melting point. 
%

We encourage more field-based research on water pocket formation and rupture mechanisms. In particular, we recommend the use of observational techniques focused on detecting and monitoring water pockets within alpine glaciers, as this lack of observation constitutes currently the main limitation for their understanding. We anticipate that our study will inspire further work in this area of research.


\section{What are the limitation of ground penetrating radar on the detection of englacial water bodies in temperate ice?}

In Chapter~\ref{ch:chapter_gprmax}, we quantified the GPR limitation in detecting englacial bodies in temperate ice by using a combination of GPR field measurements and forward modelling of the GPR signal. We found that the strong scattering and attenuation effects present in our field measurements conducted at 25\,MHz on temperate glaciers are caused by centimetres to decimeters-scale englacial water inclusions. More precisely, our numerical modeling demonstrates that even small amounts of liquid water content (e.g. 0.2,\%, which falls within the typical range for alpine glaciers), combined with decimeter-scale water inclusions, can obscure bedrock and subglacial channels reflections due to signal attenuation. In addition, we demonstrated that estimating the ice thermal regime (cold or temperate) from GPR measurements based solely on the presence of scatterers (indicative of temperate ice) or their absence (indicative of cold ice) is not straightforward. Our results showed that a clear GPR signal can actually be obtained from temperate ice containing a low amount of water inclusions. Conversely, cold ice can exhibit significant scattering and reflections from internal features (such as refrozen water and ice lenses). This complicates the interpretation of GPR data for thermal regime assessment. We encourage more field measurements that integrate GPR data with comprehensive spatial coverage of temperature measurements to advance our understanding in this matter.
%

In conclusion, while GPR is a commonly used tool in glaciology to detect englacial bodies, its success can be limited by these complex scattering effects induced by englacial water. 

\section{Perspectives}

In this thesis, outlooks are described at the end of each chapter, with additional insights in the Discussion section. Overall, I believe this work will contribute to a better understanding of glacier outburst floods stemming from ice-dammed lakes and water pockets and will inspire future research to build upon the foundation set by this thesis.


