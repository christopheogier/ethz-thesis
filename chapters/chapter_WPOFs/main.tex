\chapter{Definition, formation and rupture mechanisms of water pocket outburst floods in alpine glaciers: insights from an updated inventory for the Swiss Alps}
\label{ch:chapter_WPOFs}

\todo{reference of the preprint}
In this study I contributed to update the water pocket outburst floods inventory, performed the data analysis, produced the figures (with contribution from Dorde Masovic) and wrote the article. 

\section{Background}

\begin{itemize}


        \item  Glacier lake outburst floods (GLOFs) are typically associated with visible surface or marginal lakes (see Chapter \ref{ch:chapter_plainemorte}), leaving a significant gap in understanding events originating from hidden water reservoirs within or beneath glaciers, termed water pockets.

        \item Despite Haeberli's pioneering work on WPOFs in the Swiss Alps, defining and studying water pockets remains challenging, with ambiguity in distinguishing them from other types of glacier floods.

        \item Our study aims to fill this gap by compiling a comprehensive database of WPOFs in the Swiss Alps and analyzing their characteristics

        \item We seek to improve water pocket characterization, laying the groundwork for future research in water pocket prediction

    \end{itemize}

    %\cite{Emmer&al2022} suggest promoting GLOF trigger-focused analysis and hazard as sessments and data-driven re-analysis o GLOF susceptibility indicators -> that is what we do here for WPOFs (check redundancy with the introduction)

    %\cite{Veh&al2022}: We emphasize that it is vital to distinguish better between physical and research-oriented drivers of regional GLOF trends or realistic assessments of changing hazard potential and related and risk under future atmospheric warming. Our study answer the first need, without making speculative assumption of the second point.