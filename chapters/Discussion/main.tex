\chapter{"Global" Discussion on outburst floods and Outlook}
\label{ch:discussion}

\section{The future of glacial outburst floods research}

Here we really discuss what is the contribution of the thesis in the global scope of glacial outburst flood. In that sens this more an outlook chapter than a pure discussion (point-specific discussions are present at the end of each chapter). We take perspective and implement all the different bits of this thesis in the worls of GLOFs

% I think I should present some of the discussion/outlook bits of the chapter. For instance the discussion on some of the parameters (such as characteristic length) should be here, the relevance of GPR, what's next for water pocket: moniroting and modelling temperature. Here is the discussion chapter :)) and hop bast thesis is over

%Discussion on GLOFs:

%\cite{Carrivick&Tweed2016}: This problem leads us to make key recommendations that there needs to be accurate, full and standardised monitoring and recording of glacier floods, in particular to preferably discriminate flood volume and peak discharge at source rather than at some distance down valley. Otherwise the physical mechanisms responsible for generation of the flood are masked by the effects of channel topography on flood evolution with distance down valley.

%\cite{Emmer&al2022}: research on GLOFs include more and more co-authors (= increase in collaboration), and show 30\% of international collaboration (our WPOFs paper fits in).


% Discussion on the future occurence of GLOFs and WPOFs
% link with climate change and glacier retreat (more proglacial lakes, less ice to store WP?)
% example in role of climate and climate change (Zheng et al., 2021b)(cited in Emmer 2022)

% How would a stochastic distribution of nusselt and friction factor influence model output? would be interesting to take state of the art model and run sensitivity test on these parameters. Would be cool to use these model to back-simulate PM drainage :)

% use my data to calibrate validate outburst model!! (thesis from GAG)

% discuss Annegret Pohle paper on R channel hydraulics

% needs for sensitivity analysis with our stochastic variables: does it change considerably the modelled output? (see Weder 2010 with manning roughness, for instance)

% thermal regime is important for water pocket. Thermal regime can be characterized by GPR. DIscuss Alphubel and mont collon findings in this regard. 

%Interesting: Veh et al (2022) reported 312 (219 GLOFs) for the Swiss Alps (from 1560 to 2019), we report 92 WPOF. Hence, 30% of the glacier flood (n=312) are WPOF, so Haberli's statement that 30/40& of glacier flood are from water pocket stands. -> phenoman frequent and needs observations to be better understood

\subsection{Insights of Chapter~\ref{ch:chapter_plainemorte} for glacial outburst floods hazard mitigation}

\subsubsection{The evolution of the supra glacial channel at Plaine Morte from 2019 to 2024}

Few words to qualitatively describe the evolution of the channel

\subsubsection{Lac des Bossons, 2023 (French Alps)}

How my work was used for hazard mitigation on a real case at lac des Bossons. How our field measurements compare to the ones acquired on lac des Bossons? Discussion on the characteristic length for the Nusselt number calculation (this is a major point on the discussion about the future of GLOFs research). 

\subsection{Lac de Tignes, 2025 (French Alps)}

How my work will be used for hazard mitigation on a real case at lac de Tignes.

\section{Insights of Chapter~\ref{ch:chapter_WPOFs}}

\subsection{ The hypothetical drainage of the Rhonegletscher subglacial cavity discovered in 2022}

\begin{itemize}
    \item In 2022 we discovered a subglacial cavity at Rhoneglestcher
    \item Overview of the observations (GPS, GPR, MB stakes, DEM). For this we cite the various conference talk (and maybe the article in preparation?
    \item What if the cavity was a former water pocket (see mechanisms in Section XX in \ref{ch:chapter_WPOFs})? 
    \item We show that the drainage of the 5000m$^3$ would have gone unnoticed because of the buffer effect of the lake
    \item This highlights how many WPOFs events we potentially miss
\end{itemize}

% to quantify WP growing VS closure, the former being slower than the latter.  From Manu: "the kinetics of clossure is much higher (proportionnal to the cube of the ice pressure) than the kinetics of openning (proportionnal to the cube of the difference of ice and water pressures)"

% return period of WPOFs according to their magnitude

% Bias is a big thing in GLOFs reporting. What about WPOFs?

% we found that WPOF occur every 6 years in average since 1900, but we miss most of the events. A case study could be quantify better the real occurence of water pocket filling and bursting

% we need a monitoring of (at least) one of the WP mechanism identified in Ogier et al 2024!

% what is needed to understand GLOFs from surface lakes? highlight again the Gag's study outcomes on Bossons 

% discussion topic: 
%1) we got some better parametrization -> what's next to really well model surface glacier lake drainage? -> validation/calibration on case study
%2) we know water inclusions matter: what should we do on the field?
%3) we characterized WP, what's next? -> more monitoring, thermal regime, etc...

\subsection{ Strategies for detecting alpine water pockets}

\begin{itemize}
    \item Several projects ongoing to detect water pocket:
    \item SNMR by Laura G.
    \item Passive seismic
    \item Glacier thermal modelisation by Juliette Bonnet (IGE)
    \item Teledection technique was not used in this thesis, but has interesting potential for WP detection
    \item positive anomaly of MB time series as a favorable factor (see Tête Rousse in 1892)?
    \item combine with output on the GPR of the next section?
\end{itemize}

%early warning system:
%comMF: maybe one could also say that real-time discharge measurements in proglacial streams could at least for some WPOF mechanisms (temporary blockage of subglacial channels) be used for early warning for downstream areas (i.e. if unexpected drop or cut-off in discharge occurs), as we show that this mechanism occurs quite frequently (resp. is the most frequent WPOF mechanism that we detected)...

%Daniel: Don't ask me why I think of it only now, but here I was suddenly wondering: in which of the 4 mechanisms that we propose would this fit? I guess it is closest to "water filled crevasse", although it is not a crevasse at all. Does it mean that we need a 5th mechanism?  [Ok, later in the article I saw that, indeed, this was counted towards "water-filled crevasse"). And as I think of it: did Mauro H. (or someone else) correlate the number of WPOFs to MB? I suddenly start to like the idea that there might be "relictic lakes" buried within glaciers ;-) Possibly Ilaria could look up old aerial images and see whether there are anomalous GPR signals at locations where supraglacial lakes existed in periods of positive MB? It's av bit of a hair-pulled hypothesis, I admit (lakes would get advected by ice flow, and when doing so, they might drain anyway), but somehow, we still don't really have a clue how WPs form.
% -> yes this crossed my mind 3 years ago. 

%\subsection{Other}

%matthias: The shift from the observation by Haeberli that most events come from steep glaciers in comparison to our analyis might be due to a difference in the dominant types that has been induced by climate change - roof-collapses are increasingly observed now. For example, steep glaciers are unlikely to qualify for the roof-collapse type, whereas valley glaciers are unlikely to relate to the Tete-Rousse case (polythermal ice). The shift from the observation by Haeberli that most events come from steep glaciers in comparison to yours might be due to a difference in the dominant types that has been induced by climate change - roof-collapses are increasingly observed now.
% -> I think this a too weak statement. In fact, temporary blockage are inventoried all along the century, without a particular trend


% Some consideration as comments on the role of overdeeping in WPOFs:
%The drainage system of Findelen is mostly inefficient and distributed \citep{Swift&al2021,Iken&Bindschadler1986}. This might play a role in the occurrence and the frequency of these little outburst events. And sometimes, like in 2017, the temporary blockage is larger/longer than usual, and can accumulate larger volume (e.g. ~17 000 m3) and the ice dam break lead to an outburst, a so-called water pocket outburst flood.
%\cite{Hooke&Pohjola1994} indicate that Shreve's law breaks down near overdeepening and that water might route from subglacial conduit to englacial conduit in overdeepening as a response of the adverse slope. In adverse bed slope, the water freeze when most of the energy is stored as potential energy, and the subglacial pathway blocks. Because in steep adverse slopes, the energy form viscous dissipation is not enough to warm the water, which is needed since the melting point rises in response of the falling pressure. In addition, they suggest lenticular conduits with, importantly, local constriction along the way to support their observations. These local constriction could be potential location for water pocket (that is what Swift 2021 say between the lines of their introduction: another example of water pocket as "linked subglacial cavities of a distributed system"). 

%Note really relevant for the hydraulic barrier section:
%Note that the same topographic feature (i.e.\ a glacier surface depression, see Fig.~\ref{fig:plainemorteWP}) that is prone to subglacial water pockets formation is also prone to form moulins. For moulins, the subglacial hydraulic barrier is hydraulically connected to the glacier surface. Therefore, the water reservoir is not entirely subglacial but mostly supraglacial and at atmospheric pressure. If the hydraulic barrier is located along the subglacial drainage pathways that generally form during the melt season \citep{Fountain&Walder1998}, the water will be drained through the subglacial drainage network before the water pocket can form. In general, subglacial channels that form early in the melt season remain connected to the subglacial drainage system throughout the melt season. For subglacial channels remaining open during the whole melt season, the development of a water pocket at locations of hydraulic barriers is therefore unlikely. 

%We relate the fact that temporary blockage of subglacial channel is the WPOFs mechanism the most represented in our inventory with the fact that most of the WPOFs also occurred after a few days-long positive temperature anomaly (see Section~\ref{sec:envelop}). One can speculate that in the observed period of maximal positive temperature anomalies (i.e.\ , during the five days prior to the WPOF), the subglacial network experiences very high fluctuations in discharge. We suggest that these fluctuations can exert considerable mechanical work on the roof of conduits, potentially weakening it. When the water levels drop in the late afternoon or night, the conditions would be given for parts of the roof to collapse into the partially empty channels. This can block the channel temporarily until the next meltwater comes, builds up water pressure, and eventually causes the blocking to let go at once.


% Interesting comment from Manu: "what we have learned from Tête Rousse is that the opening of a water pocket is far much slower than a closure process when water has been drained. Therefore I understand that a filling by precipitation in hours or days  can play a role under the condition of a pre-existing cavity."


%One important thing relates to the mechanisms to store water: for (1), (2) and (3), existing voids are necessary that can be quickly filled without the need for further growth of the water reservoir when it is full of water. For (4), initial voids are not necessary, just some en-/subglacial water connection upstream the thermal barrier, and then the reservoir grows by ice creep due to high water pressure.

% \subsection{ A metric for the prevalence of low hydraulic slopes on glaciers}


% Here we present how we calculate the relative proportion of areas of low hydraulic potential gradient for all Swiss glaciers used in Section~\ref{sec:prediction} to complete \cite{Hupfer2023}.

% The map-plane gradient of the hydraulic potential $\nabla\,\psi$ is the vector
% %
% \begin{equation}
%      \nabla\,\psi = \left(\frac{\partial\,\psi}{\partial x}, \,\,\frac{\partial\,\psi}{\partial y}\right),
%      \label{eq:grad_phi_formula}
% \end{equation}
% %
% where $x$ and $y$ are the horizontal coordinates. 
% The L2-norm of the gradient of the hydraulic potential $\vert \nabla\,\psi \vert $ is equal to
% %
% \begin{equation}
%     \vert \nabla\,\psi \vert  = \sqrt{\left(\frac{\partial  \psi}{\partial x}\right)^2 + \left(\frac{\partial \psi}{\partial y}\right)^2}.
%      \label{eq:norm_grad_phi}
% \end{equation}
% The norm of the Shreve potential can then be calculated by substituting Equation~(\ref{eq:phi}) into the previous two equations.

% %
% %Given the equation above, surface slope is approximately ten times more important than bed slope to drive water flow at the glacier bed. This means that water can flow against reverse slope of overdeepening, as long as the bed slope is less than $\approx$10 times steeper than the surface slope. 

% We calculate the area for individual glaciers where $\vert \nabla\,\psi \vert $ is lower than 4\,\%. This threshold value is arbitrarily chosen after visualisation on a geospatial information system to isolate areas of low hydraulic gradient where water is more likely to accumulate. For each glaciers, we normalise this area by the glacier area. The higher this variable, the higher the relative proportion of area having a low hydraulic slope. 
% %Hypotheses behind: the biggest proportion of inefficient drainage area within a glacier, the most likely water accumulates, and thus water pocket to form. 

% The distributed map of the gradient-norm of the hydraulic potential for all Swiss glacier can be downloaded at \todo{link to ETHZ collection}.


\section{Insights of Chapter~\ref{ch:chapter_gprmax} on the use of GPR for glacial applications}


\subsection{Water pockets detection using GPR}

\begin{itemize}
    \item Suitable or not? For which strategies? 
    \item Alternatives to explain limitations in data interpretation: rock inclusions (see Ilaria's work)
    \item Few words on the potential of airborn-GPR analysis by Ilaria
\end{itemize}

\subsection{GPR in cold alpine ice caps}

Fieldwork at Mont Collon and Alphubel using high frequencies (cite Appendix Chueboden for method) -> let's compare our measurement with temperature from borehole. Impact of the geometry for GPR interpretations (Mt Collon vs Alphubel). See \cite{Brown&al2009} for claim on "clear ice does not necessarily mean cold ice", we support this. 

\subsection{Two applications of GPR forward modelling prior to challenging field measurements}

After the publication of \citep{Ogier&al2023}, I co-developed with Barthélémy Anhorn an user-friendly python-based package to simulate the GPR signal in various context REF URL. The aim was to extend the use of gprMax to a broader choice of geometries and material than the original gprMax package \citep{Warren&al2016}. Our package was used in AICHELE and ANNA REF. These two applications highlight the value of forward modelling in applied geophysics, when the geophysicist face a complex and challenging measurements in remote locations and want to perform cheap simulations to aquire prior knowledge (in our examples: a alpine glacier and the moon). 

\subsubsection{The use of GPR forward modelling for basal stick-slip detection}

Glacial sliding is a major component of glaciers dynamic, and its understanding is crucial for accurate projection of sea level rise caused by ice sheet melting. Glacier sliding has shown seismic evidence of stick-slip motion similar to fault locking. Locating and assessing the extent of stick-slip remains challenging due to low seismic amplitudes \citep{Graff&al2021}. The ruptures can occur entirely in the till, debris or ice layer, or at interfaces between the bedrock/till and the ice.
The dominant frictional mechanisms can be various as by viscoelastic till deformation, by velocity strengthening on hard beds, or by rate-and-state friction, but none of these mechanisms is known to prevail in which domain.

Current in situ monitoring methods rely on boreholes equipped with seismic or GPS sensors, tilt sensors, and cameras, and have limitations. A non-invasive in situ imaging method would be ideal for locating and tracking stick-slip events in space and time with minimal disturbance to the medium.

Imaging by non invasive methods can be performed by optical and ultrasonic methods. Both methods utilize similar principles: rapid imaging via either light or ultrasound waves determinate the relative displacement between two consecutive "snap-shots", and they use speckles as target to track displacement. Speckle are small (i.e. sub-wavelength) scatterers that are randomly distributed. 

GPR as tool - phase correlation

In a novelty approach, REF Aichele used the water inclusions I modelled in \cite{Ogier&al2023} as speckles for synthetic imaging of a stick slip event.

Highlight my contribution

Is it a discussion, or an outlook?



\begin{itemize}
    \item \textbf{Importance of Understanding Glacier Movement:} Understanding glacier movement is crucial for predicting sea-level rise, climate change, and freshwater supply, as glaciers are dynamic bodies that deform internally and slide at their base.
    \item \textbf{Significance of Basal Sliding:} Basal sliding, characterized by rapid spatio-temporal dynamics, is a key aspect of glacier movement. However, uncertainties in the physics of basal sliding hinder accurate predictions of glacier sliding rates and, consequently, future ice-sheet behavior and climate models.
    \item \textbf{Proposal for Unraveling Basal Sliding Physics:} The proposal aims to unravel the physics of basal sliding by directly observing basal processes through a novel imaging approach.
    \item \textbf{Focus on Glacier Stick-Slip:} The study focuses on glacier stick-slip, a localized accumulation and release of stress similar to earthquake ruptures, which plays a crucial role in basal sliding.
    \item \textbf{Limitations of Current Observation Methods:} Existing observation methods, such as seismology and borehole measurements, have limitations, including low seismic amplitudes and the need for prior knowledge of slip location.
    \item \textbf{Proposed Imaging Method:} The proposed method involves non-invasive, in situ measurements of glacier stick-slip using phase-sensitive Ground Penetrating Radar (GPR) to resolve basal motion.
    \item \textbf{Advancement from Laboratory to Field:} While direct rupture imaging has been limited to laboratory settings, the proposal aims to take this imaging technique to the field, specifically onto alpine glaciers.
    \item \textbf{Exploiting Radar Speckle:} The study aims to exploit radar speckle, the diffuse backscattering of radar waves inside glaciers, for in situ displacement imaging.
    \item \textbf{Novelty of Approach:} The proposed method of in situ speckle tracking and monitoring dynamic phenomena like stick-slip with GPR is innovative and has not been previously attempted.
    \item My contribution in the 2D forward simulation ("Proof of concept").
    The glaciers models input used in \cite{Ogier&al2023} were used to simulate the stick slip in basal sliding. To do so, we considered the water inclusions as markers for ice deformation, i.e. we created a synthetic stick slip by shifting all water inclusions along a 45\,° plan, with a Gaussian-gradient like in the displacement amplitude. This displacement is meant to reproduce a simplistic ice deformation at the bed after a stick slip event. FIGURES. The resulted displacement field (FIGURE) served a a synthetic observations to test the phase-change tracking algorithm of AICHELE (REF).

    DON't Go too much in detail-> I "just" provided the tool.
    Are results promising? What the expected challenges for a real field application?
\end{itemize}








Questions pour johannes: we consider gprmax as a APRES, right?

What are the limitation: until which fraction of the wavelenght we can in theory track displacements?

Low frequency: good penetration but low accuracy. High res means high accuracy but less penetration. Is that correct when applied for phase change tracking?


\subsubsection{The use of GPR forward modelling prior to field measurements on the moon}

\begin{itemize}
    \item My contribution on the moon-simulation project of Anita Moldenhauer
\end{itemize}



