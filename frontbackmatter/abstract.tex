%*******************************************************
% Abstract
%*******************************************************
%\renewcommand{\abstractname}{Abstract}
\pdfbookmark[1]{Abstract}{Abstract}
\begingroup
\let\clearpage\relax
\let\cleardoublepage\relax
\let\cleardoublepage\relax

\chapter*{Abstract}


Glacial lakes can form in various locations relative to glaciers, when the resulting meltwater is naturally dammed by ice, moraine, or bedrock. When the lake dam fails, or is overtopped by the lake water, the lake drainage can lead to an outburst flood, so called glacial lake outburst floods (GLOFs). If the outburst flood stem from a water pocket, which is a reservoir of water impounded within the glacier, we call it water pocket outburst flood (WPOF). GLOFs and WPOFs pose significant hazards to downstream infrastructures and communities, and can cause severe geomorphic changes. Understanding GLOFs and WPOFs is crucial for modelling and thus predicting their impacts. However, current modeling of glacial outburst floods is hindered by significant uncertainties due to the lack of in situ field measurements during drainage events, as well as limited data on the reservoirs themselves in the case of water pockets. This thesis aims to improve the understanding of glacier outburst floods, in particular stemming from ice-dammed marginal lakes and water pockets, and to improve the interpretation of ground penetrating radar (GPR) signals for detecting englacial water pocket in alpine glaciers. The thesis is organized into three chapters.
%

The first chapter presents extensive field measurements from the ice-dammed lake drainage into a supraglacial channel at Glacier de la Plaine Morte (Switzerland). We studied the hydraulics and thermodynamics parameters that control the channel geometry and the channel stream flow during the lake drainage. In particular, we calculated the Darcy-Weisbach friction factor and the Nusselt number that are used in modeling studies of ice-dammed lake drainage. The Nusselt number controls the channel melt rate and thus the lake outflow discharge. Our results show that the Darcy-Weisbach friction factor vary from 0.17 to 0.48 during the drainage and should be considered as a stochastic variable in modelling studies. We found that the Nusselt number differs significantly according to which method is used, and propose new empirical coefficients to calculate the Nusselt number based on our observations. We suggest to consider these empirical coefficient as stochastic variables too. We recommend future research to develop a unified modeling framework for ice-dammed lake drainage into supraglacial channel based on comprehensive field data.
%

The second chapter characterizes the spatial and temporal distribution of WPOFs in the Swiss Alps and investigates the drivers behind their formation and rupture. Through an updated and extended inventory of 91 WPOFs from 37 glaciers since 1699, we analyzed the glacio-geomorphic and meteorological drivers for 32 of these events since 1961. Based on our inventory, we suggest that smaller-scale topographic and glacio-geomorphic variables control the WPOF occurrence, rather than glacier-wide glacio-geomorphic variables. We show that high temperature within few days prior to the WPOF and strong precipitations during the day of the outburst explain most of the events. This suggest that rapid (within few days) water accumulation from melting and rainfall triggered most WPOFs reported in our inventory. Based on our inventory and a literature review, we identified four main mechanisms for water pockets formation: temporary blockage of subglacial channels, hydraulic barriers, water-filled crevasses, and thermal barriers (i.e. formation of water pockets at the transition of cold and temperate ice). We encourage more field-based research aiming at monitoring water pockets to advance understanding of WPOFs.
%

When it comes to detect features under the glacier surface (such as water pockets), ground penetrating radar (GPR) is generally used. The third chapter discuss the limitation in detecting englacial bodies in temperate glacier by using GPR. We combined field measurements and numerical modeling at 25\,MHz, revealing that englacial water inclusions cause strong scattering and attenuation effects. We demonstrated that even small amounts of liquid water content (e.g., 0.2\,\%) with decimeter-scale water inclusions can obscure bedrock reflections. These findings highlight the challenges in interpreting GPR signals of englacial water bodies in temperate ice, and also in interpreting the glacier thermal regime (cold or temperate) based on the presence or not of scatterers. Indeed, we show that temperate ice can appears free of scatterers for low liquid water content, and cold ice can show scatterers from internal features. 
%

We believe this thesis to motivate future work aiming at modelling ice-dammed lake drainage, understanding outburst floods from water pockets, and improving detection of englacial features in ice using GPR.



\endgroup

\clearpage

\begingroup
\let\clearpage\relax
\let\cleardoublepage\relax
\let\cleardoublepage\relax

%\begin{otherlanguage}{nfrench}
\pdfbookmark[1]{Résumé}{Résumé}
\chapter*{Résumé}

Résumé en francais ici.

%\end{otherlanguage}

\endgroup

\vfill