%*******************************************************
% Abstract
%*******************************************************
%\renewcommand{\abstractname}{Abstract}
\pdfbookmark[1]{Abstract}{Abstract}
\begingroup
\let\clearpage\relax
\let\cleardoublepage\relax
\let\cleardoublepage\relax

\chapter*{Abstract}


Glacial lakes can form in various locations relative to glaciers, when the resulting meltwater is naturally dammed by ice, moraine, or bedrock. When the lake dam fails, or is overtopped by the lake water, the lake drainage can lead to an outburst flood, so called glacial lake outburst floods (GLOFs). If the outburst flood stem from a reservoir of water within the glacier, we call it water pocket outburst flood (WPOF). GLOFs and WPOFs pose significant hazards to downstream infrastructures and communities, and can cause severe geomorphic changes. Understanding GLOFs and WPOFs is crucial for modelling and thus predicting their impacts. However, current modeling of glacial outburst floods is hindered by significant uncertainties due to the lack of in situ field measurements during drainage events, as well as limited data on the reservoirs themselves in the case of water pockets. This thesis aims at improving both the understanding of glacier outburst floods, in particular stemming from ice-dammed marginal lakes and water pockets, and the interpretation of ground penetrating radar (GPR) signals for detecting englacial water pocket in alpine glaciers. The thesis is organized into three chapters.
%

The first chapter presents extensive field measurements from an ice-dammed lake drainage into a supraglacial channel at Glacier de la Plaine Morte (Switzerland). We studied the hydraulic and thermodynamic parameters that control the channel geometry and the channel stream flow during the lake drainage. In particular, we calculated the Darcy-Weisbach friction factor which characterize the channel flow resistance, and the Nusselt number, which controls the channel melt rate and thus the lake outflow discharge. Theses two parameters are used in modeling studies of ice-dammed lake drainage. Our results show that the Darcy-Weisbach friction factor vary from 0.17 to 0.48 during the drainage and should be considered as a stochastic variable in modelling studies. We found that the Nusselt number differs significantly depending on which method for calculation is used, and propose new empirical coefficients to calculate the Nusselt number based on our observations. We suggest to consider these empirical coefficient as stochastic variables too. We recommend future research to develop a unified modeling framework for ice-dammed lake drainage into supraglacial channel based on comprehensive field data. This framework could integrate various independent field measurements, including those presented in this thesis, to allow a robust comparison of recent lake drainage models.
%


The second chapter characterizes the spatial and temporal distribution of WPOFs in the Swiss Alps and proposes mechanism for their formation and rupture. Through an updated and extended inventory of 91 WPOFs from 37 glaciers since 1699, we analyzed the glacio-geomorphic and meteorological drivers for 32 of these events since 1961. Based on our inventory, we suggest that smaller-scale topographic and glacio-geomorphic variables control the WPOF occurrence, rather than glacier-wide glacio-geomorphic variables. We show that high temperature within few days prior to the WPOF and strong precipitations during the day of the outburst coincide with most of the events. This suggest that rapid (within few days) water accumulation from melting and rainfall may have triggered most WPOFs reported in our inventory. Based on our inventory and a literature review, we propose four main mechanisms for water pockets formation: temporary blockage of subglacial channels, hydraulic barriers, water-filled crevasses, and thermal barriers (i.e. formation of water pockets at the transition of cold and temperate ice). We encourage more field-based research aiming at detecting and monitoring water pockets to advance understanding of WPOFs.
%

In the case of englacial features detection, such as water pockets for example, such field-based efforts can make use of ground penetrating radar. The third chapter thus explores the limitation the limitation in detecting englacial bodies in temperate glacier by using GPR. We combined field measurements and numerical modeling at 25\,MHz, revealing that englacial water inclusions cause strong scattering and attenuation effects in GPR signals. We demonstrated that even small liquid water content (e.g., 0.2\,\%) with decimeter-scale water inclusions can obscure bedrock reflections. These findings highlight the challenges in interpreting GPR signals of englacial water bodies in temperate ice, and also in interpreting the glacier thermal regime (cold or temperate) based on the presence or not of scatterers. Indeed, we show that temperate ice can appears free of scatterers for low liquid water content.
%

Overall, this thesis provides i) an unique field-based dataset useful for future work aiming at modeling ice-dammed lake drainage, ii) a better understanding of water pocket outburst floods, and iii) improves the interpretation of detected englacial features in ice using GPR.



\endgroup

\clearpage

\begingroup
\let\clearpage\relax
\let\cleardoublepage\relax
\let\cleardoublepage\relax

%\begin{otherlanguage}{nfrench}
\pdfbookmark[1]{Résumé}{Résumé}
\chapter*{Résumé}

Les lacs glaciaires peuvent se former à divers endroits par rapport aux glaciers, lorsque l'eau de fonte résultante est naturellement barrée par la glace, les moraines ou le lit rocheux. Lorsque ce barrage cède ou est submergé par l'eau du lac, le drainage du lac peut entraîner une inondation soudaine appelée débâcle de lac glaciaire (en anglais GLOFs, acronyme pour glacier lake outburst floods). Si l'inondation provient d'un réservoir d'eau situé dans le glacier, on parle alors de débâcle par poche d'eau (en anglais WPOF, acronyme pour water pocket outburst floods). Les GLOFs et les WPOFs posent des risques important pour les infrastructures et les communautés en aval, et peuvent provoquer des changements géomorphologiques du paysage. Comprendre les GLOFs et les WPOFs est crucial pour la modélisation de ces événements et ainsi pour prédire leurs impacts. Cependant, la modélisation actuelle des inondations glaciaires est limitée par d'importantes incertitudes en raison du manque de mesures de terrain in situ pendant les événements de drainage, ainsi que des données limitées sur les réservoirs eux-mêmes dans le cas des poches d'eau. Cette thèse vise à améliorer à la fois la compréhension des inondations glaciaires, en particulier celles issues des lacs barrés par la glace et des poches d'eau, et l'interprétation des signaux de radar à pénétration de sol (en anglais GPR, acronyme pour ground penetrating radar) pour détecter les poches d'eau à l'intérieur des glaciers alpins. La thèse est organisée en trois chapitres.
%

Le premier chapitre présente les mesures de terrain étendu d'un drainage de lac glaciaire dans un canal de surverse au Glacier de la Plaine Morte (Suisse). Nous avons étudié les paramètres hydrauliques et thermodynamiques qui contrôlent la géométrie du canal et l'écoulement d'eau dans le canal pendant le drainage du lac. En particulier, nous avons calculé le facteur de friction de Darcy-Weisbach qui caractérise la résistance à l'écoulement du canal, et le nombre de Nusselt, qui contrôle le taux de fonte du canal et donc le débit de sortie du lac. Ces deux paramètres sont utilisés dans les études de modélisation du drainage des lacs glaciaires. Nos résultats montrent que le facteur de friction de Darcy-Weisbach varie de 0,17 à 0,48 pendant le drainage et devrait donc être considéré comme une variable stochastique dans les études de modélisation. Nous avons constaté que le nombre de Nusselt diffère considérablement en fonction de la méthode de calcul utilisée, et nous proposons de nouveaux coefficients empiriques pour calculer le nombre de Nusselt basés sur nos observations. Nous suggérons de considérer également ces coefficients empiriques comme des variables stochastiques. Nous recommandons aux futures recherches de développer un cadre de modélisation unifié pour le drainage des lacs glaciaires dans un canal de surverse basé sur des données de terrain complètes. Ce cadre pourrait intégrer diverses mesures de terrain indépendantes, y compris celles présentées dans cette thèse, pour permettre une comparaison robuste des modèles récents de drainage des lacs.
%

Le deuxième chapitre caractérise la distribution spatiale et temporelle des WPOFs dans les Alpes suisses et propose des mécanismes possible pour leur formation et leur rupture. À travers un inventaire mis à jour et étendu de 91 WPOFs provenant de 37 glaciers depuis 1699, nous avons analysé les facteurs glacio-géomorphologiques et météorologiques pour 32 de ces événements depuis 1961. Basés sur notre inventaire, nous suggérons que des variables topographiques et glacio-géomorphologiques à petite échelle contrôlent l'occurrence des WPOFs, plutôt que des variables glacio-géomorphologiques à l'échelle du glacier. Nous montrons que des températures élevées dans les jours précédant les WPOFs et de fortes précipitations le jour de l'évènement coïncident avec la plupart des rupture de poches d'eau. Cela suggère que l'accumulation rapide d'eau (en quelques jours) due à la fonte et aux précipitations a pu déclencher la plupart des WPOFs rapportés dans notre inventaire. Basé sur notre inventaire et une étude de la littérature, nous proposons quatre mécanismes principaux pour la formation des poches d'eau par: blocage temporaire des canaux sous-glaciaires, barrières hydrauliques, crevasses remplies d'eau et barrières thermiques (c'est-à-dire formation de poches d'eau à la transition de la glace froide et tempérée). Nous encourageons davantage de recherches sur le terrain visant à détecter et surveiller les poches d'eau pour faire progresser la compréhension des WPOFs.
%

Dans le cas de la détection d'objets intra-glaciaire, telles que les poches d'eau par exemple, de tels efforts basés sur le terrain peuvent utiliser le GPR. Le troisième chapitre explore donc les limitations du GPR dans la détection des objets intra-glaciaire dans les glaciers tempérés. Nous avons combiné des mesures sur le terrain et une modélisation numérique associées à une fréquence de 25\,MHz, révélant que les inclusions d'eau liquide situées dans la glace causent des effets de diffusion et d'atténuation du signal GPR importants. Nous avons démontré que même une faible teneur en eau liquide (par exemple, 0,2\,\%) avec des inclusions d'eau de taille décimétrique peut obscurcir les réflexions du lit rocheux. Ces résultats mettent en évidence les défis liés à l'interprétation des signaux GPR des objets intra-glaciaire dans la glace tempérée, ainsi qu'à l'interprétation du régime thermique du glacier (froid ou tempéré) en fonction de la présence ou non de sources de diffusion dans le signal GPR. En effet, nous montrons que la glace tempérée peut sembler exempte de sources de diffusion pour une faible teneur en eau liquide.

Dans l'ensemble, cette thèse fournit un ensemble de données de terrain unique qui sera utile aux travaux futurs visant à modéliser le drainage des lacs glaciaires barrés par la glace, à mieux comprendre les débâcles glaciaire par rupture de poche d'eau, et à améliorer l'interprétation des objets intra-glaciaire détectés dans la glace à l'aide du GPR.

%\end{otherlanguage}

\endgroup

\vfill